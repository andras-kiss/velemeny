%############################################# ANDRÁS KISS ##########################################
%################################################ 2018 ##############################################
\documentclass[a4paper, 11pt, oneside, bibliography=totoc]{article}
\def\magyarOptions{hyphenation=huhyphn}
\usepackage{ae,aecompl}
\usepackage[T1]{fontenc}
\usepackage[utf8]{inputenc}
\usepackage[hungarian]{babel}
\usepackage{indentfirst}
\usepackage{xymtex}
\usepackage{multirow}
\usepackage{gensymb}
\usepackage{upgreek}
\usepackage[geometry]{ifsym}
\usepackage{subfig}
\usepackage[version=3]{mhchem}
\usepackage{float}
\usepackage{textcomp}
\frenchspacing
\usepackage[dvips]{graphicx}
\usepackage{color}
\usepackage{anysize}
\marginsize{3.2cm}{2.8cm}{3cm}{2cm}
\usepackage{enumerate}
\usepackage{cite}
\usepackage{listings}
\usepackage{setspace}
\usepackage{marginnote}
\setstretch{1.2}
\usepackage{xcolor}

\begin{document}
\title{Témavezetői vélemény Szili Szilárd \\ \textit{,,A Belouszov--Zsabotyinszkij reakció pásztázó elektrokémiai mikroszkópos térképezése''} \\ című szakdolgozatáról}

\maketitle

Szilárd 2017-ben kezdett témavezetésem mellett foglalkozni az elektrokémiai mikroszkóppal. Tanszékünk elektrokémiai kutatócsoportja főleg mikroskálán megfigyelhető jelenségek elektroanalitikai vizsgálatával foglalkozik, így Szilárd az ehhez szükséges mikroelektródok készítésével és karakterizálásával kezdte a munkát. A 2017 június elejétől augusztus végéig terjedő nyári gyakorlata során ezen kívül megismerkedett az alapvető elektroanalitikai módszerekkel is. Szilárdnak eredetileg egy egyszerű, már viszonylag jól vizsgált témát jelöltem ki. Az antimon mikroelektróddal történő pH térképezés további lehetőségeit és akadályait kezdte vizsgálni. A munkát rendkívül nagy fegyelemmel és szorgalommal végezte. Nyári gyakorlata során nyílvánvalóvá vált számomra, hogy Szilárd ennél jóval többre lesz képes, és a választottnál érdekesebb, egyben nagyobb kihívást nyújtó témát érdemel. Ezért megkérdeztem Szilárdtól, hogy szeretne-e egy olyan kutatáshoz csatlakozni, mely a pásztázó elektrokémiai mikroszkópot egy teljesen új módon használja oldatfázis felületi pásztázására. Szilárd nagy lelkesedéssel kezdett dolgozni az új problémán. Az irodalomban erre még nem volt példa, a téma nekem is új volt. Mivel ritka jelenség, hogy homogén oldatfázisban mintázat alakul ki, a például szolgáló rendszer kiválasztása könnyű volt. Az kémiai oszcilláló jelenségek közül a már nagyon jól jellemzett Belouszov-Zsabotyinszkij oszcilláló reakciót (BZ-reakció) választottam. Szilárd a BZ-reakcióban kialakuló redox-potenciál mintázatot térképezte pásztázó elektrokémiai mikroszkóppal. 2017 szeptembere és 2018 áprilisa között végezte a kísérletes munkát. Az erről, körülbelül három hónap leforgása alatt írt szakdolgozata a hivatkozásjegyzékkel együtt 32 számozott oldalból áll, így megfelel a terjedelemmel szemben támasztott követelményeknek. A dolgozat arányos felépítésű, nagy része körülbelül egyenlően van felosztva a három nagy fejezetre (Irodalom, Módszerek, Eredmények). A Bevezetés kellő távolságból közelít rá a választott témára. A Célkitűzések fejezet világosan, pontokba szedve írja le a munka célját. A dolgozat irodalomjegyzéke igényes, egységes formátumú, hibát csak elvétve találni (, pl. [27]). Dícséretes, hogy a dolgozat tizenöt ábrája közül tizennégy Szilárd sajátkészítésű ábrája. A dolgozat LaTeX szövegformázó rendszerrel készült, éppen ezért tipográfiailag helyes, és esztétikus. Néhány ábra azonban rossz betűméretet használ, például a 4.6 ábra betűi túl nagyok, az 5.2 és 4.5 betűi túl kicsik. A LaTeX előnyei mellett egyik nagy hátránya az automatikus helyesírás ellenőrzés hiánya, és ez sajnos meg is látszik a dolgozaton. Viszonylag sok az elgépelés. A dolgozat végső verziójának produkálása és nyomtatása előtt egy alaposabb ellenőrzésre lett volna szükség. Így valószínűleg elkerülhetőek lettek volna a 3.3 és 3.4 egyenletben lévő sztöchiometriai hibák is. A felsorolt hibák mind formai jellegűek, az egyenletekben vétett hibák pedig természetesen csak pillanatnyi figyelmetlenség eredményei. A dolgozat nyelvezete jó, de helyenként nehézkes a fogalmazás. Szilárdnak mindenképpen javítania kell az írásbeli kifejezőképességén.

A kitűzött célokat végül sikerült elérni. Ezt nagy teljesítménynek tartom Szilárd részéről, hiszen nem csupán egy már jól ismert témának egy kis részletén dolgozott, hanem a kutatócsoportunkban használt technika egy teljesen új alkalmazását dolgozta ki témavezetésem mellett. Meggyőződtem róla, hogy Szilárd a kísérletes munka és a szakdolgozat írása során alkalmazott módszerek minden részletét érti, azokat magas szinten elsajátította. A dolgozat messzemenően kielégíti a BSc. szakdolgozattal szemben támasztott követelményeket. A védésre a következő kérdéseket teszem fel:

\begin{enumerate}
\item A kísérleti paraméterek milyen változtatása szükséges élő rendszerekben megjelenő kémiai hullámok, kémiai mintázatok térképezéséhez?
\item Vizsgálta-e a BZ-reakciót kisebb skálán, esetleg kevesebb, vagy több dimenzióban? Ha igen, milyen tapasztalatai vannak?
\end{enumerate}

Sikeres védés esetén javaslom a jeles érdemjegy odaítélését.

Kiss András

Egyetemi tanársegéd

Általános és Fizikai Kémia Tanszék

Kémiai Intézet

Pécsi Tudományegyetem


2018, június 16

Homburg (Saar)

Németország



\end{document}
