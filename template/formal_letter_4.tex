%%%%%%%%%%%%%%%%%%%%%%%%%%%%%%%%%%%%%%%%%
% Professional Formal Letter
% LaTeX Template
% Version 2.0 (12/2/17)
%
% This template originates from:
% http://www.LaTeXTemplates.com
%
% Authors:
% Brian Moses
% Vel (vel@LaTeXTemplates.com)
%
% License:
% CC BY-NC-SA 3.0 (http://creativecommons.org/licenses/by-nc-sa/3.0/)
%
%%%%%%%%%%%%%%%%%%%%%%%%%%%%%%%%%%%%%%%%%

%----------------------------------------------------------------------------------------
%	PACKAGES AND OTHER DOCUMENT CONFIGURATIONS
%----------------------------------------------------------------------------------------

\documentclass[11pt, a4paper]{letter} % Set the font size (10pt, 11pt and 12pt) and paper size (letterpaper, a4paper, etc)

\input{structure.tex} % Include the file that specifies the document structure

%\longindentation=0pt % Un-commenting this line will push the closing "Sincerely," and date to the left of the page

%----------------------------------------------------------------------------------------
%	YOUR INFORMATION
%----------------------------------------------------------------------------------------

\Who{Kiss András} % Your name

\Title{, PhD} % Your title, leave blank for no title

\authordetails{
	Ált. és Fiz. Kém. Tsz.\\ % Your department/institution
	Ifjúság útja 6.\\ % Your address
	Pécs, 7624\\ % Your city, zip code, country, etc
	akiss@gamma.ttk.pte.hu\\ % Your email address
	Belső mellék: 61021\\ % Your phone number
}

%----------------------------------------------------------------------------------------
%	HEADER CONTENTS
%----------------------------------------------------------------------------------------

\logo{pte_logo.eps} % Logo filename, your logo should have square dimensions (i.e. roughly the same width and height), if it does not, you will need to adjust spacing within the HEADER STRUCTURE block in structure.tex (read the comments carefully!)

\headerlineone{PÉCSI} % Top header line, leave blank if you only want the bottom line

\headerlinetwo{TUDOMÁNYEGYETEM} % Bottom header line

%----------------------------------------------------------------------------------------

\begin{document}

%----------------------------------------------------------------------------------------
%	TO ADDRESS
%----------------------------------------------------------------------------------------

\begin{letter}{
%	Prof. Jones\\
%	Mathematics Search Committee\\
%	Department of Mathematics\\
%	University of California\\
%	Berkeley, California 12345
}

%----------------------------------------------------------------------------------------
%	LETTER CONTENT
%----------------------------------------------------------------------------------------

\opening{
\begin{center}Témavezetői vélemény Szili Szilárd \emph{,,A Belouszov--Zsabotyinszkij reakció pásztázó elektrokémiai mikroszkópos térképezése''} című szakdolgozatáról
\end{center}
}

%\flushleft

Szilárd 2017-ben kezdett témavezetésem mellett foglalkozni az elektrokémiai mikroszkóppal. Tanszékünk elektrokémiai kutatócsoportja főleg mikroskálán megfigyelhető jelenségek elektroanalitikai vizsgálatával foglalkozik, így Szilárd az ehhez szükséges mikroelektródok készítésével és karakterizálásával kezdte a munkát. A 2017 június elejétől augusztus végéig terjedő nyári gyakorlata során ezen kívül megismerkedett az alapvető elektroanalitikai módszerekkel is. Szilárdnak eredetileg egy egyszerű, már viszonylag jól vizsgált témát jelöltem ki. Az antimon mikroelektróddal történő pH térképezés további lehetőségeit és akadályait kezdte vizsgálni. A munkát rendkívül nagy fegyelemmel és szorgalommal végezte. Nyári gyakorlata során nyílvánvalóvá vált számomra, hogy Szilárd ennél jóval többre lesz képes, és a választottnál érdekesebb, egyben nagyobb kihívást nyújtó témát érdemel. Ezért megkérdeztem Szilárdtól, hogy szeretne-e egy olyan kutatáshoz csatlakozni, mely a pásztázó elektrokémiai mikroszkópot egy teljesen új módon használja oldatfázis felületi pásztázására. Szilárd nagy lelkesedéssel kezdett dolgozni az új problémán. Az irodalomban erre még nem volt példa, a téma nekem is új volt. Mivel ritka jelenség, hogy homogén oldatfázisban mintázat alakul ki, a például szolgáló rendszer kiválasztása könnyű volt. Az kémiai oszcilláló jelenségek közül a már nagyon jól jellemzett Belouszov-Zsabotyinszkij oszcilláló reakciót (BZ-reakció) választottam. Szilárd a BZ-reakcióban kialakuló redox-potenciál mintázatot térképezte pásztázó elektrokémiai mikroszkóppal. 2017 szeptembere és 2018 áprilisa között végezte a kísérletes munkát. Az erről, körülbelül három hónap leforgása alatt írt szakdolgozata a hivatkozásjegyzékkel együtt 32 számozott oldalból áll, így megfelel a terjedelemmel szemben támasztott követelményeknek. A dolgozat arányos felépítésű, nagy része körülbelül egyenlően van felosztva a három nagy fejezetre (Irodalom, Módszerek, Eredmények). A Bevezetés kellő távolságból közelít rá a választott témára. A Célkitűzések fejezet világosan, pontokba szedve írja le a munka célját. A dolgozat irodalomjegyzéke igényes, egységes formátumú, hibát csak elvétve találni (, pl. [27]). Dícséretes, hogy a dolgozat tizenöt ábrája közül tizennégy Szilárd sajátkészítésű ábrája. A dolgozat LaTeX szövegformázó rendszerrel készült, éppen ezért tipográfiailag helyes, és esztétikus. Néhány ábra azonban rossz betűméretet használ, például a 4.6 ábra betűi túl nagyok, az 5.2 és 4.5 betűi túl kicsik. A LaTeX előnyei mellett egyik nagy hátránya az automatikus helyesírás ellenőrzés hiánya, és ez sajnos meg is látszik a dolgozaton. Viszonylag sok az elgépelés. A dolgozat végső verziójának produkálása és nyomtatása előtt egy alaposabb ellenőrzésre lett volna szükség. Így valószínűleg elkerülhetőek lettek volna a 3.3 és 3.4 egyenletben lévő sztöchiometriai hibák is. A felsorolt hibák mind formai jellegűek, az egyenletekben vétett hibák pedig természetesen csak pillanatnyi figyelmetlenség eredményei. A dolgozat nyelvezete jó, de helyenként nehézkes a fogalmazás. Szilárdnak mindenképpen javítania kell az írásbeli kifejezőképességén.

A kitűzött célokat végül sikerült elérni. Ezt nagy teljesítménynek tartom Szilárd részéről, hiszen nem csupán egy már jól ismert témának egy kis részletén dolgozott, hanem a kutatócsoportunkban használt technika egy teljesen új alkalmazását dolgozta ki témavezetésem mellett. Meggyőződtem róla, hogy Szilárd a kísérletes munka és a szakdolgozat írása során alkalmazott módszerek minden részletét érti, azokat magas szinten elsajátította. A dolgozat messzemenően kielégíti a BSc. szakdolgozattal szemben támasztott követelményeket. A védésre a következő kérdéseket teszem fel:

\begin{enumerate}
\item A kísérleti paraméterek milyen változtatása szükséges élő rendszerekben megjelenő kémiai hullámok, kémiai mintázatok térképezéséhez?
\item Vizsgálta-e a BZ-reakciót kisebb skálán, esetleg kevesebb, vagy több dimenzióban? Ha igen, milyen tapasztalatai vannak?
\end{enumerate}

\begin{center}
\textbf{Sikeres védés esetén javaslom a jeles érdemjegy odaítélését.}
\end{center}

Kiss András
Egyetemi tanársegéd
Általános és Fizikai Kémia Tanszék
Kémiai Intézet
Pécsi Tudományegyetem


2018, június 16
Homburg (Saar)
Németország

%\closing{Sincerely,}

%----------------------------------------------------------------------------------------
%	OPTIONAL FOOTNOTE
%----------------------------------------------------------------------------------------

% Uncomment the 4 lines below to print a footnote with custom text
%\def\thefootnote{Témavezetői vélemény}
%\def\footnoterule{\hrule}
%\footnotetext{\hspace*{\fill}{\footnotesize\em Footnote text}}
%\def\thefootnote{\arabic{footnote}}

%----------------------------------------------------------------------------------------

\end{letter}

\end{document}
